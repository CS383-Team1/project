\documentclass[12pt]{article}
\usepackage{geometry}
\usepackage{titling}

\newcommand{\subtitle}[1]{
	\posttitle{
		\par\end{center}
		\begin{center}\large#1\end{center}
		\vskip0.5em
	}
}

\geometry{margin=1in}

\title{Statistics}
\subtitle{CS383 - Team \texttt{\textasciicircum teamname}}
\date{\today}
\author{
Tessa Saul
}

\begin{document}

\maketitle

\section*{Characters}
\subsection*{Player Characters}
Player characters have several core statistics that are involved in combat. 
These include:
\begin{itemize}
\item Health
\item Mana
\item Action Points 
\end{itemize}
The first is a statistic that describes the character's health, and 
ability to continue in the fight. If this statistic drops to zero, the 
character can no longer fight. This statistic is usually called health-points,
abbreviated to hp. I believe Lance mentioned calling it ``Will to Live'', or 
``Will to Go On'', which I think fits nicely with our depressing office 
setting.

The next is a representation of the character's energy to perform special 
actions, like big, fancy attacks. I believe Mason suggested calling this 
statistic ``caffeine''. Once again, this fits nicely into the setting. 

Lastly, action points are a measure of how many actions a character can take
in a round of combat. A player can break up their action points into several 
actions, or use most of them for a single more powerful action.  

\section*{Non-Player Characters}
Non-Player character have health, mana, and action points like player 
characters, but they will also have some personality statistics. These may
include:
\begin{itemize}
\item Friendliness
\item Irritability 
\item Immune System 
\end{itemize}
Friendliness will determine how much reputation increasing actions affect a
certain NPC. Irritability can influence how much reputation decreasing actions 
affect an NPC. Immune system can determine whether an NPC is in the office, 
if we choose to make this a feature. 

\section*{Items}
\subsection*{Weapons}
Weapons all have three statistics which are 
\begin{itemize}
\item Base Power
\item Accuracy 
\item Chance to Critical Hit
\end{itemize}
\subsection*{Clothing}
Clothing and accessories modify a player's base statistics, and the statistics 
of their weapons. 
\subsection*{Consumables}
Consumable items include items like snack, drinks, and medicine. Consumables 
can give temporary buffs, or instant effects, like restoring health. 

\section*{Attack Types}
There will be three types of attacks:
\begin{itemize}
\item Verbal
\item Virtual 
\item Physical
\end{itemize}
\subsection*{Verbal}
There will be several types of verbal attacks which include 
\begin{itemize}
\item Compliment 
\item Insult 
\item Small Talk 
\item Logic or Technical 
\end{itemize}
All of these attack types can affect a character's accuracy, will to go on, 
health, and so on. Characters will be more susceptible or less susceptible
 to certain attack types based on their equipment. 
 
 We could have a single technical type, or many like accounting and IT. 
 
 Players will generally have access to base versions of the first three verbal
 attacks, but the weapon you are holding may give you better versions of these,
 and may provide access to a technical attack. 
  \subsection*{Virtual}
  For simplicity of play, a character's items may be digitized when they enter 
  virtual reality, so they may use the same weapons. Virtual enemies, however,
  may not be susceptible to verbal attacks. 
  
 \subsection*{Physical}
 There will be two types of physical attacks. These are 
 \begin{itemize}
 \item Bludgeoning 
 \item Ranged
 \end{itemize}
 Physical attacks will be used against aliens, but not against humans.

\end{document}