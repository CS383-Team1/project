\documentclass[12pt]{article}
\usepackage{geometry}
\usepackage{titling}

\newcommand{\subtitle}[1]{
	\posttitle{
		\par\end{center}
		\begin{center}\large#1\end{center}
		\vskip0.5em
	}
}

\geometry{margin=1in}

\title{Mechanics}
\subtitle{CS383 - Team \texttt{\textasciicircum teamname}}
\date{\today}
\author{
Tessa Saul
\and Tyler Wittreich
}

\begin{document}

\maketitle

\begin{section}{Combat}
Our combat system will be JRPG style.  The character will run into enemies in
the map and be transitioned
into a battle mode. The battle mode will be used for combat and conversations.
From there, characters can choose their actions for each turn.
\end{section}

\begin{section}{Turns}
Turns are used to regulate play and denote the passage of time.  In our game,
turns will represent the passage of time, but will not represent a realistic
passage of time.

Our turns will be represented by ticks in game play where default
actions will be performed if no other action is chosen.  In this way, all
players will contribute to the actions of a single turn. Each tick, characters
will perform actions, like attacking,
regaining health, speaking, and making save checks.

A mechanism for determining which character's actions are performed
first in a turn will exist, based on character statistics like how much
caffeine they have in their system.
\end{section}

\begin{section}{Action Points}
Characters will have some form of action points which control how many actions
the character can perform each turn in combat. Actions
that can be regulated by action points include attacks, attempts to flee, item
usage, and special actions.  Whatever we call this in the game, it is merely a
way to control how many actions, can be taken each turn.

There are a few things we should decide about how to regulate actions.
\begin{itemize}
\item Is it possible for characters to gain more actions?
\item Do characters gain more actions at higher levels.
\item Do characters gain more actions with better equipment.
\item What are the relative costs of different actions
\item Are there actions that every character always gets, such as speaking, or changing weapon?
\end{itemize}

There will be action points when it comes to conversation and combat.
While in a combat or conversation turn base state, a character will get to
choose one move or gesture. In combat this could be to attack, run and use an
item. In conversation this could be to insult, compliment, or ask a coworker if
he needs help. The amount of action points that a character has should be kept
the same throughout the game, while only increasing the characters stats and
items.
\end{section}

\begin{section}{Catch Up}
The game will get progressively harder for the character to progress to higher
and higher levels. The difficulty of combat, acquisition of key items, and
reputation requirements will become harder across the three main levels of the
game. The office levels will be easy. The cyber levels will be mediocre and the
space station levels will be the hardest to beat.
\end{section}

\begin{section}{Dice}
While in combat a character can choose to roll a dice which if he gets a certain
number is able to flee the attacker without damage. This will not kill the
attacker, but allow your character to build up more stats in the free world if
you realize you are not strong enough to beat a certain attacker. The dice will
be based off an algorithm that takes a ratio of the attacker’s stats and your
stats, giving you certain odds to roll certain numbers on a dice.
\end{section}

\begin{section}{Movement}
The character will move in a four direction tile formation. In the real-time
state the tile movement will be very small making it seem as if the character
can move in more than four directions. The character will be bounded by the
level map and environmental obstacles that are place on the map. Certain items
that a character acquires can eliminate certain environmental obstacles. In a
turn base state his movements will be limited by the actions he selects from his
character menu while in conversation or combat.
\end{section}

\begin{section}{Resource Management}
A mechanic of a lot of games is resource management.  This involves regulating
and replenishing resources like ammo, coffee, health items, and other expendable
resources.  Players can replenish their resources from sources such as
co-workers, other players, vending machines, and the refrigerator.

In resource management characters will also have the ability to manage NPC
followers. Characters can choose to make NPC friends stop following them.
Characters could also choose to give NPC followers items to increase their
loyalty.
\end{section}

\begin{section}{Worker Placement}
A mechanic of some games is worker placement. This involves
directing workers to perform a tasks while the player does something else.

If we implement this, it will probably take the form of directing interns or
secretaries to do paperwork, bring the character coffee, or spread gossip about
characters.

Once a character gains enough reputation, he will gain the ability to ask
coworker to follow him. If a character is required to perform a big takes, he
can delegate his followers to help him with the work or increase his reputation
from his followers talking to other NPC’s
\end{section}

\begin{section}{Saving}
The game will have checkpoints throughout it where the player can save.  Players
can teleport among checkpoints. They can also trade with other players at
checkpoints, and can be resurrected at checkpoints if they die.

Depending on the size of the quest, task, or level there will be a limited
number of times a character can save until he completes that task, quest, or
level.
\end{section}

\begin{section}{Risk and reward}
In certain stages of the game you will have to achieve promotions and if your
character decides to ask for a promotion and he doesn’t have enough reputation,
he might get it based off of an odds algorithm, or he will be denied. If denied
a promotion, a character will lose a substantial amount of his reputation. The
reputation advancement of the game is based off user actions that might help or
hurt his reputation depending on what he chooses.
\end{section}

\begin{section}{Winning}
Winning in this game will involve reaching the end of the game, and possibly
defeating an end boss. After a player wins, the game will be open world for them
to explore, find hidden areas, and collect awesome gear.

The main storyline of the game is to get to the end and defeat the final boss.
The game will have a percentage of completion that can only be 100 percent
complete if a character has achieved maximum stats. Once this is completed a
certain door will open in the game and the character will get the ultimate item
or unlock a cheat menu.
\end{section}

\begin{section}{Losing}
The game will have checkpoints periodically throughout. If a player dies in the
game, they will be resurrected at the last checkpoint they
checked in at in the state they checked in. The player may also lose
a small amount of reputation when they die. It is noted that this penalty must
be small enough so that they game is still fun, even with frequent deaths.
\end{section}

\begin{section}{Trading}
The game will allow trading between players, but this can only be done
at checkpoints where saving can be done, to prevent duplication of items.

There will be different levels of trading areas that characters can access. This
will be based on the characters stats, to prevent to unfair trading of extremely
strong items to newly created characters.
\end{section}

\begin{section}{Character Creation}
Character Creation will consist of filling out a job application type form, with
details for the character's physical appearance, like hair and eye color.  This
may include a preview of the character being created, so players can see
their avatar before committing. This feature is included for the story.

The different color of suit that the character picks changes the characters
primary attributes. Certain colors of suits will allow for cheaper stats
increase compared to non-primary stats.
\end{section}

\begin{section}{Victory condition}
The game will be based on goals and loss avoidance. The character will try to
complete goals that lead to new levels and untimely the final boss. On the
characters journey he will also have to avoid death by combated or termination
because of low reputation.
\end{section}

\begin{section}{In-Game Time}
This section is a wishlist item.

The game will include an in-game clock. There will be an in-game work week
for which different events happen at certain times. These events could include:
\begin{itemize}
	\item vending machine refills
	\item NPCs calling in sick
	\item special NPCs presence or absence
\end{itemize}

Another feature of the in-game time is dreamland levels for players who
do not wish to go to work at night.
\end{section}

\begin{section}{Other}
We plan to have the main quest line for our game take 7-10 hours for completion.
\end{section}

\begin{section}{Behavioral Momentum}
The character will have a default action in combat and conversation if the
player is taking too long. This behavior momentum will be based off a universal
set of default actions that each class of player has.
\end{section}

\begin{section}{Bonus}
A player will receive a bonus in terms of stats upgrades or special items
depending on what task, level, or quest he completes.
\end{section}

\begin{section}{Cascading information}
The character will not know the entire game story at once. He will only get the
appropriate amount of information to understand the narrative of the gameplay
progression.
\end{section}

\begin{section}{Community Collaboration}
Players will be able to use their acquired NPC friends or other willful online
players to complete certain tasks.
\end{section}

\begin{section}{Countdown}
The only countdown will be during a players turn in a turn base state. A player
will be given an appropriate amount of time to make a move. If he does not, the
characters behavioral momentum will perform the task.
\end{section}

\begin{section}{Discovery}
Players will be able to explorer the level maps to find new items, collect
money, and build reputation by discovering new coworkers.
\end{section}

\begin{section}{Meaning}
The game will begin by trying to find a job. After working at a new company you
noticed the coworkers are impoverished which enlightens you to begin your quest
to find and seize the corruption at the top of the corporate ladder.
\end{section}

\begin{section}{Free Gift}
If a player is building good reputation quickly within the office, he might
receive a free gift/item from another coworker.
\end{section}

\begin{section}{Levels}
The game will consist of three main types of levels. The first levels will be
the office levels. During these levels a player must gain a certain amount of
positive reputation to advance to the next set of levels. The next set of levels
will be the cyber levels. A player will have to complete certain tasks that will
destroy the company’s cyber security, which opens previously secured doors. This
allows advanced to the next set of levels, the space station levels. During
these levels you will engage in combat and try to escape the alien jail where
you have been tricked into getting locked into. After completion you will have
completed the characters main meaning of the game stated above.
\end{section}

\begin{section}{Ownership}
A player will acquire items throughout the game that that character owns. A
character can also have control over other coworkers if the player has a certain
amount of reputation.
\end{section}

\begin{section}{Points}
Money will be the game currency. A character can acquire money through
exploration, combat, conversation, trade, or task completion. A character owns
his money and can choose to spend it at his discretion. Money can be used to buy
items, increase stats, or unlock quests.
\end{section}

\begin{section}{Progression}
A player will be able to view his game progression. He can view the overall game
progression based on level completion. He can also view his stats, quests,
tasks, exploration, and item acquisition progression.
\end{section}

\begin{section}{Quests}
These will be challenges that a player will attempt to complete. For each quest,
there will be a time restraint that the player must complete the quest under.
For each quest completion, a player will gain reputation, stats, items, or
money.
\end{section}

\begin{section}{Status}
A player will have a status called a job title. There are many status that a
player can achieve. A player will achieve a new job title through increased
reputation and quest/task completion.
\end{section}

\end{document}
