\documentclass[12pt]{article}
\usepackage{geometry}
\usepackage{graphicx}
\usepackage{titling}

\newcommand{\subtitle}[1]{
        \posttitle{
                \par\end{center}
                \begin{center}\large#1\end{center}
                \vskip0.5em}
}

\geometry{margin=1in}

\setcounter{tocdepth}{2}

\title{Requirements Document}
\subtitle{CS 383 - Team \texttt{\textasciicircum teamname}}
\date{\today}
\author{
Abdulmajeed Alotaibi - \textit{alot4458@vandals.uidaho.edu} \\
\and Casey Blair - \textit{blai1919@vandals.uidaho.edu} \\
\and Mason Fabel - \textit{fabe0940@vandals.uidaho.edu} \\
\and Arthur Putnam - \textit{putn0266@vandals.uidaho.edu} \\
\and Robin Rakowski - \textit{rako4161@vandals.uidaho.edu} \\
\and Ronald Rodriguez - \textit{rodr3849@vandals.uidaho.edu} \\
\and Tessa Saul - \textit{saul7812@vandals.uidaho.edu} \\
\and Lance Wells - \textit{well3112@vandals.uidaho.edu} \\
\and Lance Wells - \textit{witt0241@vandals.uidaho.edu} \\
}

\begin{document}
\maketitle

\section{Functional Requirements}
\begin{itemize}
	\item \textbf{Startup}
	
	The game will start with front screen with picture for the game and under
	it three choices: A) New Game, B) Load Game, and C) Join Game. The player
	will be able to choose any of the three choice by clicking on it.

	\begin{itemize}
	\item New Game

	New Game will be a fresh start for the player and it will start look like
	it is the first time for the player play the game even if the player
	already played the game and saved the game but want to play again from the
	beginning, that will be possible. Moreover, the player will get a message
	which is system reports domainname/IP, so other players can join him.

	\item Load Game

	When the player click on load game it will take the player to last place
	that the player saved before quit the game. Moreover the system will
	reports domainname/IP for the player as message so other player can join in
	the adventurer.

	\item Join Game

	One of the choices the player will be able to chose is join game auto join,
	the server will check any other player who does not have partner, if it
	find it will match them.
	\end{itemize}
	
	\emph{Description by Abdulmajeed Alotaibi}
	
	\item \textbf{Drop-in Drop-Out Multiplayer}
	
	 When users start the game, they will be asked if they want to join an existing game hosted by another user. New users can join and leave other users’ games without the need for all users to start a new multiplayer game. This allows
for a more fluid gaming experience as new users can come and go as they wish without interrupting other users’ progress. And with up to 27 users joining one game, this will be necessary to have a fluid gameplay experience that isn’t constantly being interrupted by new users wanting to
join.

\emph{Description by Casey Blair}
\end{itemize}

\section{Functional Requirements}
\begin{itemize}
	\item \textbf{Single/Multi-player}
	
	Plays standalone or multiplayer, scaling up to 27 players
	
	\item \textbf{Server Functionality}
	
	Peer-to-peer connection to a "server" thread allows remote player
	interaction with the application via a proxy thread
	
	\item \textbf{Dungeon Naming} 
	
	As this game is going to be set in a Sci-fi world, it would make sense, on 
	initial thought, to erase the idea of using the term "dungeon" to describe 
	the environment one is in during a mission, and instead referring to it as 
	a "site" or a "facility" or something of that nature. We believe, however, 
	that it may be too early in the planning stages of the game to make this 
	decision definitively. We may very well end up having some "themed" levels 
	that take place inside of a computer that could most definitely be referred 
	to as dungeon-esque. 
	
	\emph{Description by Ronald Rodriguez}
	
	\item \textbf{Level Editor}
	
	Use whatever internal data structure(s) your software design calls
	for, but the external file format of levels is: ASCII-art text; i.e.
	Emacs or Vi can be your level editor.
	
	\item \textbf{Setting 1}
	
	The game starts at the Spaceport America, a 28 square mile
	complex located in the Jornada del Muerto region in New Mexico, the
	center of modern commercial space travel. Your levels may
	start with approximations of the major buildings, starting with the more
	office/administrative buildings and working towards the space operations
	buildings.
	
	\item \textbf{Setting 2}
	
	Levels 1 to $k$ are relatively ordinary "office" training levels. There is
	no death in the initial levels; the emphasis is on social standing,
	promotion, bureaucracy, verbal jousting, maybe the occasional photobomb,
	copy machine prank, toner theft, etc. See the Wikipedia article on
	verbal self defense. NPC friends that you make here, if you gain enough
	faction with them, will follow and aid you semi-autonomously. Interface
	is text/ASCII-art, roguelike. 
	
	\item \textbf{Setting 3}
	
	Levels $k+1$ to $m$ are "cyber" levels, with death causing the player to bedumped out of cyberspace.
	Tron-like. Emphasis is on knowledge discovery, resouce acquisition,
	virus combat. Interface is tile-based, 2D graphical, nethacklike.
	Real-time until you hit "engagements" by entering enemy
	zones-of-control; then turn-based.
	
	\item \textbf{Setting 4}
	
	Levels $m+1$ to $n$ are "space station" levels, with death=game over.
	Emphasis is on combat with alien weapons and/or computer cyberattacks.
	Interface is tile-based, enhanced (or 2.5D?) graphical. 
\end{itemize}

\section{Rejected Requirements}
\begin{itemize}
	\item \textbf{Player Threads}
	
	System views each player as a thread, thread sends/receives messages
	to a "controller" thread that manipulates game state.
	
	It was originally proposed that the system will view each player as a
	thread, allowing those threads to send and receive messages from a
	controller thread which manages the game state. This requirement has been
	rejected for being too specific. While this is a good solution to the
	general gameplay requirements, we believe it is better to keep this as a
	suggestion rather than a requirement.
	
	\emph{Description by Mason Fabel}
		
	\item \textbf{3-D Gameplay}
	
	We said tile-based, not 3D FPS. If you want to do 3D in 480/481 or
	428, knock the 2D version out of the park 
	
	It was mentioned that expanding this project to include 3D graphics might
	be a future possibility. We believe that designing this project in such a
	way to intentionally leave this path open is a distraction from the main
	purpose of the game, and thus this should not be a requirement.
	
	\emph{Description by Mason Fabel}
	
\end{itemize}
\end{document}
